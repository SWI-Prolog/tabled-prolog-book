\chapter{Introduction to First-Order Logic} \label{fol}

This chapter introduces the basics of logic needed to understand the
foundations of Prolog and logic programming.  We start with
Propositional Logic and then move to first-order logic.

\begin{verbatim}
 Real World Situation                 Statements
--------------------               -------------------
|Pic of person      |--------------|''a red block is |
|looking at a table |              | on the table''  |
|with blocks on it  |          /   -------------------
|with balloon saying|   person
``a red block is on |    with
the table'' and     |     L    \   What statements are about
another person (with|              ----------------
|L on chest) looking|--------------|  Table with  |
|at them            |              |     blocks   |
--------------------               ---------------
\end{verbatim}
Fig 11

Figure 11 shows in cartoon form what logic tries to do.  On the left
is a real-world situation of a person looking at a table with blocks
on it and saying that ``a red block is on the table''.  There ia
another person, our erstwhile logician, looking at this situation and
trying to understand what's going on with the use of language.  The
logician separates this mixed-up real-world situation into several
parts to study them.  First she pulls herself out of the situation so
she can study it ``from outside''.  Second she separates the statement
into one component and the part of the real world that the statement
is about into another component.  Her idea is that a statement is true
or false when looked at in the context of the real-world component.
For example, when we look at the table with the blocks, we will see
that either the statement ``a red block is on the table'' is a true
statement about that situation, i.e., there is indeed a block on the
table whose color is red; or it is a false statement about that
situation, i.e., looking at the blocks and table, we see that none of
the blocks are red.  So the meaning of a statement with respect to a
real-world situation is either {\em true} or {\em false}, depending on
whether the statement makes a true or false assection about that
world.  This is called a correspondence theory of meaning, since
statements ``mean'' true or false depending on how they ``correspond''
with a world situation.

As an aside, note that there are other possible theories of language
meaning.  As of this writing, there is a US presidential nominating
process going on.  It is rather clear to me that the language used in
that process does not depend on a correspondence theory of meaning.  A
suprisingly large number of utterances and clearly not true with
respect to the world situation, so claiming that politicians make
statements that are true is certainly not an adequate account of
language in that context.  Another theory of language, going by the
name of ``speech-acts'', models statements as efforts by one person to
cause another person to construct in his head certain beliefs.  I
suspect political discourse is much better modeled by such a theory of
language.  Of course there is the question of what a belief is, and
the formaliztion of that may require some correspondence theory of a
belief and the world.  End of aside.

Returning to our logician (whose role we now play), the problem is how
to formalize the language and the world situation in such a way that
we can precisely say when a sentence is true (or false) in a world
situation.  To do this, we need to precisely formalize the language
component, which is called the syntax of the system, and the real
world component, which is call the semantics of the system.  And then
we have to define a function that takes a sentence in the syntax and
a world situation in the semantics, and produces either true or false.
This is the meaning (or semantic) function of the framework.

After we do this, another question arises: Can we deduce the truth
values of some sentences given knowledge of truth values of other
sentences?  This is a question of deduction.  Knowing that some
sentences are true an a given situation, can we deduce that some other
sentence is true in that same sitution?  And can we do this without
knowing what the situation is?  A very simple (yet powerful) example
of this is the following:  Say we know that we are in a situation in
which the sentence ``it is raining'' is true; furthermore we know that
in all situations the sentence ``If it is raining, then the grass is
wet'' is true.  Then we should be able to conclude that we are in a
situation in which the sentence ``the grass is wet'' is true.  And we
know this entirely from the form of the syntax of the sentences, and
their meanings with respect to their truth values in situations.  This
is an example of deduction, and we want the logical system of meaning
we construct to support such reasoning.

To summarize, a logic (or logical theory of meaning) has three
components: 1) a syntax that defines the well-formed sentences of the
language of the logic, 2) a model-theoretic semantics that defines
what models of real-world situations are appropriate for the logic
language and how those models make the well-formed sentences true or
false, and 3) a theory of deduction that gives rules that apply to
sentences that preserve some aspects of their truth.  We will look
first at propositional logic and its components, and then at
first-order logic.

\section{Propositional Logic}

Propositional logic takes simple declarative sentences as basic and
combines them with sentential operators: ``and'', ``or'', and ``not''
(and perhaps others).  For example, we might have basic sentences like
``John is going to the store'', ``It is raining'', and ``It is
daytime''.  Propositional logic has basic connective words: ``and'',
``or'', and ``not''.  We can combine the basic sentences to make
larger sentences using the basic connective words.  For example, we
could say: ``John is going to the store and it is daytime'', or ``It
is not raining or it is daytime''.

\subsection{Syntax}

In propositional logic we are interested is the connective
words, and not particularly in how meaning is given to the basic
declarative sentences.  So we will simply represent the basic
sentences by letters: p,q,r,s,....  We construct more complex
sentences using the connectives to get sentences like: not(p), p or q,
p or (q and not(p)), etc.

So the syntax of propositional logic is very simple, and can be given
by the following grammar:

\begin{verbatim}
Sent --> Conjsent
Sent --> Sent or Conjsent
Conjsent --> Primesent
Conjsent --> Conjsent and Primesent
Primesent --> Basicsent
Primesent --> not Primesent
Basicsent --> Prop_sym
Basicsent --> ( Sent )
\end{verbatim}

\subsection{Semantics}

The meaning of a sentence in a world is a truth value: true or false.
I.e., given a world, each sentence is either true or false in that
world.  So we will think of a world as assigning to each basic
sentence a truth value, i.e., true or false.  I.e., a world determines
a truth assignment to the propositional symbols (which represent the
basic sentences).  For example for the basic sentences (propositions)
given above, ``It is raining'' and ``It is daytime'', one world might
assign ``true'' to the first and ``false'' to the second.  This would
be the world in which it is a wet night.  A given set of $n$
propositional symbols can distinguish $2^n$ different worlds, one for
each truth assignment to that set of propositional symbols.

Given the meanings of the propositional symbols, we want to give
meanings to the compound sentences, those made up from basics
propositional symbols using the propositional connectives.  So let P
be the set of proposition symbols, and $M:P~ \rightarrow
~\{true,false\}$ be a truth assignment (associated with some possible
world; i.e., we think of $M$ as representing a possible world.)  We
can extend $M$ to the set of all sentences, i.e., the compound
sentences involving the connectives, as follows:

\begin{verbatim}
M(and(S1,S2)) = true, if M(S1)=true and M(S2)=true,
              = false, otherwise

M(or(S1,S2)) = true, if M(S1)=true or M(S2)=true, or both
              = false, otherwise

M(not(S)) = true, if M(S)=false
          = false, otherwise.
\end{verbatim}

Given a propositional sentence, we canconsider its truth value in each
possible world, i.e., for each truth assignment to the propositional
symbols in the sentence.  For example, consider (not p or q).

\begin{verbatim}
p     | q     || not p or q
=======================
true  | true  || true
true  | false || false
false | true  || true
false | false || true
\end{verbatim}

Here we have set out all four of the truth assignments (worlds) for
the basic propositional symbols, p and q, one in each row.  And then
in the third column, we have entered the corresponding truth value of
the compound sentence (not p or q) in that world.
This is known as a truth table for (not p or q), and we can make a
truth table for any (complex) sentence in the propositional logic.  If
the sentence has $n$ different propositional variables, its truth
table will have $n+1$ columns and $2^n$ rows.

A sentence whose truth table has a last column of all {\em true} is
called a {\bf tautology}.  One whose last column is all {\em false} is
called a {\bf contradiction}.  A sentence that is a contradiction is
said to be {\bf unsatisfiable} since no world can satisfy it, i.e.,
make it true.  For example, (b or not b) is a tautology.  And (p and
not p) is a contradiction.  Notice that if an arbitrary sentence Q is
a tautology, then (not (Q)) is a contradiction.

Tautologies are of particular interest.  Say we know that (not (A) or
B) is a tautology, for some particular sentences A and B.  Say further
that we don't know everything about our current world, but we do know
that it makes A true.  Then we also know that our current world,
whatever it is, makes B true.  (Exercise, why?)  We often write a
formula of the form (not(A) or B) as $(A \rightarrow B)$, read A
implies B, or if A then B.  We can extend the logic with this
implication operator as a new connective.

\subsection{Deduction}

So to be able to reason about what is true in our current world,
without know everything about the world, it is useful to know the
tautologies.  Now we can always determine if a sentence is a tautology
by building its truth table.  But a truth table is exponentially
large.  Is there a way to determine whether a sentence is a tautology
without having to construct the entire truth table, just by looking at
the sentence itself and maybe related sentences?

There are many different ways to tackle this problem.  We will look at
one particular approach.  First we will consider sentences that have
identical truth tables; such sentences are called {\bf equivalent},
indicated by $\Longleftrightarrow$.  There are a number of important
equivalences; some useful ones are listed here:
\begin{enumerate}
\item $A~ \Longleftrightarrow ~not~ not~ A$
\item $A ~or (B~ and~ C)~ \Longleftrightarrow ~(A~ or~ B)~ and~ (A~ or~ C)$
\item $not(A~and~B)~\Longleftrightarrow ~not(A)~or~not(B)$
\item $not(A~ or~ B)~ \Longleftrightarrow ~not(A)~ and~ not(B)$
\end{enumerate}
Now given any sentence, we can use these equivalences, iteratively, to
transform it into an equivalent sentence of the following form
(called conjuctive normal form): all ``not'' operators are immediately
over propositional symbols, and no ``or'' is inside the scope of an ``and.
For example, not(A and (B or not(C))), can be transformed as follows:

$not(A~and~(B~or~not(C))) $ \\
$\Longleftrightarrow not(A)~or~not(B~or~not(C))$ \\
$\Longleftrightarrow not(A)~or~(not(B)~and~not(not(C)))$ \\
$\Longleftrightarrow not(A)~or~(not(B)~and~C)$ \\
$\Longleftrightarrow (not(A)~or~not(B))~and~(not(A)~or~C)$

We call a propositional symbol or the negation of a propositional
symbol a ``literal''.  We can see that, for {\em any} sentence, we can
find an equivalent sentence that is a conjunction of disjunctions of
literals.  We can think of a sentence in this conjunctive normal form
as a set of ``clauses'', one clause for each conjunct, and each clause
consisting of a set of literals.  So for the example above, not(A and
(B or not(C))) is equivalent to the set of clauses
$\{\{not(A),not(B)\},\{not(A),not(C)\}\}$.  A set of clauses is true if all
the clauses are true (corresponding to the conjunctions in the
conjunctive normal form) and a single clause is true if any one (or
more) of the literals in it is true (corresponding to the disjunction
in the equivalent sentence.)  A literal is true in the expected cases:
if it is a propositional symbol then it is true just in case the
propositional symbol is true; if it is the not of a propositional
symbol, then it is true just in case the propositional symbol is
false.

Now we will present a method (called {\bf resolution}) to determine
whether a set of clauses is a contradiction, i.e. is unsatisfiable.
That means that every row of its truth table is false.  Given such a
method, we can use it to determine whether an arbitrary sentence is a
tautology, as follows.  We take the formula, put a {\em not} around it,
convert it to an equivalent sentence in conjunctive normal form (using
the equivalences above) and then apply resolution, which will tell us
whether it is a contradiction.  If it is, then the original sentence
is a tautology.

Note that any clause that has both a literal and its complement is
satisfied by every truth assignment.  Thus we can delete such a clause
(a tautology) from any set of clauses without affecting that set's
unsatisfiability.  From now on we will assume no clause in a clause
set contains both a literal and its complement.

As mentioned resolution works on clauses.  The idea is, given a set of
clauses, to choose two from the set that can be resolved, form their
{\bf resolvant}, which is a new clause, and add the resolvant to the
original set of clauses.  And continue doing this until the empty
clause is generated, or until no new clauses can be generated.  If the
empty clause is generated, then the original set of clauses is a
contradiction.

So we need to define the ``resolvant'' of two clauses.  Consider two
distinct clauses, one of which contains a proposition symbol P and the
other of which contains the literal $not(P)$. These clauses can be
resolved and the resolvant is all the literals in either of the two
clauses excluding the literals $P$ and $not(P)$.  For example, the two
clauses: $\{p,not(q)\}$ and $\{not(p),not(q),r\}$, can be resolved on $p$ in the
first clause and $not(p)$ in the second, producing the resolvant:
$\{not(q),r\}$.  Notice that we treat the clauses as sets and thus don't
have multiple copies of a literal in a clause.

Do a larger example:

\subsubsection{Soundness}
If we are given a set of clauses that is satisfiable, i.e. for which
there exists a truth assignment that makes them all true, then if we
form a resolvant from that set and add it back to the set, that new
set of clauses is also satisfiable.  This follows because consider the
truth assignment that makes the entire set of clauses true; it must
make both the two clauses that we resolved true.  And it must make the
propositional symbol we resolved on either true or false.  If it makes
it true, then it makes its negation false.  Then for the clause
containing the negation to be true, some literal in it other than that
negation must be true.  That literal is in the resolvant and so the
resolvant is true.  Alternatively the satisfying truth assignment
might make the proposition symbol we resolve on false.  If so, then
some other literal in its clause must be true (since the clause is
true.)  And that literal will be in the resolvant making the resolvant
true.  So adding resolvants preserves satisfiability of a set of
clauses.  If we eventually can get the empty clause, which is
unsatisfiable, then the original set of clauses must be
unsatisfiable.  (If you're not clear on why the empty clause must be
unsatisfiable, you can back up one step, and see that the only way we
can generate the empty clause through resolution is by resolving 
two singleton clauses of the form $\{p\}$ and $\{not(p)\}$.  An it's easy to
see that no truth assignment can make both these clauses true.)
So this argument shows that our tautology testing procedure is
{\bf sound}, i.e., if it says a sentence is a tautology, then it is
indeed a tautology.

\subsubsection{Completeness}
But we would also like for our procedure to be {\bf complete}, i.e., if
indeed the original sentence is a tautology, then our procedure will
indeed tell us so.  It turns out that resolution is indeed complete.
The proof is not difficult so we will look at it.

Consider an arbitrary sentence.  Let $p1, p2, p3, ..., pn$ be all the
proposition symbols that appear in it.  Consider the following tree
built from these symbols:

\begin{verbatim}
                      []
                 /            \
           ~p                       p
         /     \                  /    \
     ~q           q           ~q          q
    /   \        /  \        /   \       /  \
  ~r     r    ~r    r      ~r     r    ~r    r
  /\     /\   /\    /\     /\    /\    /\    /\
 ~s s  ~s s  ~s s  ~s s   ~s s  ~s s  ~s s  ~s s 
\end{verbatim}

(We note that this tree is {\em not} an execution tree of any kind.
It is not used for computation in any way.)

Each path from the root to a leaf corresponds to one truth assignment
to the proposition symbols $\{p,q,r,s\}$, the truth assignment that
makes all the literal on the path true.

Now consider the following example clauses:
\begin{enumerate}
\item $\{p, r\}$
\item $\{\sim p, r, s\}$
\item $\{\sim q, \sim r\}$
\item $\{q, r\}$
\item $\{\sim p, \sim s\}$
\item $\{q, \sim r, s\}$
\item $\{p, q, \sim r\}$
\end{enumerate}

We can use each clause to eliminate various truth assignments
represented in the tree as not satisfying the clause (and thus not
satisfying the set of clauses.)

For each clause, find all paths from the root down to a node such that
the complement of every literal in the clause is in the path (and this
is not true for any shorter path.)  Mark all such nodes with an 'X'.
Do that for every clause in the clause set.  Note that any root to
leaf path (i.e. truth assignment) that has an X on some node cannot
satisfy the set of clauses, because it will make the clause that
generated the X false.  And note conversely that any root-to-leaf path
that does {\em not} have a node marked with an X will make all the
clauses true and so is a satisfying truth assignment.  So an
unsatisfiable set of clauses will have a tree in which every
root-to-leaf path contains a marked node.  Consider the subtree from
the root down to the first marked node on each path, i.e., a subtree
with all its leaves marked and no internal nodes marked.  In this
tree, there must be a node with both children being leaves.  Let the
children correspond to not(P) and P for some P.  These child nodes
were caused to be marked by two clauses, one of which contains not(P)
and one of which contains P.  These two clauses are resolvable, and
will produce a new resolvant clause that would cause some ancestor of
these two nodes to be marked.

These observations show that the resolution closure of a set of
unsatisfiable clauses must contain the empty clause, the only clause
that marks the root of the tree.  If it didn't, there would be a
resolution that could be done, and thus the set wouldn't be closed.

So the reolution strategy for determining unsatisfiability is sound
and complete.

Resolution is highly explosive, there being many different choices for
what clauses to resolve and on what literals to resolve them at every
step.  It turns out that there are a variety of restrictions one can
place on the order of generating resolvants that reduce the
combinatorial explosion and yet retain the soundness and completeness
of the method.  Much research has gone into finding such constraints.
Another direction for finding efficient deduction strategies has been
to restrict the form of the allowable clauses in some way.  With a
restricted set of clause forms, and restricted ways to apply
resolution, efficient strategies do sometimes exist.

\subsection{Horn Clauses}

In this section we consider clauses of a particular form, called Horn
clauses.  A clause is a Horn clause if it contains at most one
positive literal.  So the following are examples of Horn clauses:
\begin{verbatim}
{~p,q,~r,~s}
{~p}
{s}
{~p,q}
{~p,~q,~r,~s}
{}
\end{verbatim}
and the following are {\em not} Horn clauses:
\begin{verbatim}
{~p,q,~r,s}
{s,t}
{p,~q,~r,~s,u}
\end{verbatim}
A Horn clause that does contain a positive literal is called a {\em
Definite Horn Clause} or just a {\em Definite Clause}.  So the first,
third and fourth example Horn clauses above are definite clauses.  

Now consider resolution among a set of Horn clauses.  For two clauses
to resolve, one must contain a positive literal, i.e. one must be a
definite clause.  Also note that the resolvant of two Horn clauses is
itself a Horn clause, so Horn clauses are closed under resolution.
Also note that resolving two definite clauses together can never
directly produce the empty clause, since one (and exactly one) of the
positive literals will appear in the resolvant.  One way to do
resolution with Horn clauses is to start with a nondefinite clause and
resolve it with a definite clause to generate another nondefinite
clause.  Then we use that just-generated nondefinite clause and
resolve it against some definite clause to get a new nondefinite
clause, and we continue in this way, always using the just-generated
resolvant as one clause used to generate the next resolvant.  This is
called a linear strategy.  It turns out that this is also a complete
strategy for Horn clauses.

Consider the example, where for convenience we put the positive
literal of definite clauses first:
\begin{verbatim}
1. {~p,~r}
2. {p, ~t}
3. {p, ~q, ~r, ~s}
4. {q, ~r}
5. {r}
6. {s, ~q}.
\end{verbatim}

Consider the following resolution tree:
\begin{verbatim}
                   1. {~p,~r}
              w/2. /    \ w/3.               
              {~t,~r}   {~q,~r,~s}
                          |w/4.
                        {~r,~s}
                          |w/5.
                         {~s}
                          |w/6.
                         {~q}
                          |w/4.
                         {~r}
                          |w/5.
                          {}
\end{verbatim}
This is a tree in which the root is a nondefinite clause and all the
nodes are labeled with resolvants and each edge is labeled with the
number of the clause that the parent resolved with to generate the
child.

Notice that we kept the order of the clauses in generating the
resolvant and put the negative literals of the definite clause before
the remaining negative literals of the nondefinite clause (eliminating
subsequent duplicates.)  And we always chose the first literal in the
nondefinite clause to resolve on.  This is called SLD resolution for
Horn clauses, and it turns out to be sound and complete.  (A proof is
beyond what I want to do here, but can be found in the literature, or
reconstructed by looking an all possible resolution trees and
determining that any general resolution sequence that leads to the
empty clause, can be transformed into one that is in the form here.)

There is another way to write Horn clauses; we can write them as
implications.  We can define implication (\verb|->|), read
``implies'', by:
\begin{verbatim}
P -> Q == ~P \/ Q
\end{verbatim}
and reverse implication
(\verb|->|), read ``if'', by:
\begin{verbatim}
P <- Q == P \/ ~Q
\end{verbatim}
Note that the propositional sentence \verb|p <- (q/\r/\s)| is
logically equivalent to \verb|(p\/~q\/~r\/~s|, and is equivalent to
the Horn clause:
\begin{verbatim}
{p,~q,~r,~s}
\end{verbatim}
So definite Horn clauses are implications in which a conjunction of
positive literals imply a single positive literal.

Explain (and give example) how SLD resolution (not eliminating
redundant literals) is what Prolog does (for predicates without any
arguments).  Prolog searches the tree of SLD resolution derivations in
depth-first order.

\subsubsection{Horn Clauses, Unit Resolution}

Another strategy for resolution with Horn clauses is called unit
resolution.  In unit resolution every resolution step involves one
clause that has only one positive literal.

So for example:
\begin{verbatim}
0. {~p,~s}
1. {p,~q,~r}
2. {p,~s}
3. {q,~s,~t}
4. {r}
5. {s}
\end{verbatim}
and the following unit resolutions:
\begin{verbatim}
6. 1 and 4 --> {p,~q}
7. 3 and 5 --> {q,~t}
8. 2 and 5 --> {p}
9. 0 and 5 --> {~p}
10. 8 and 9 --> {}
\end{verbatim}
It turns out that unit resolution is complete for Horn clauses.  

We think of the indefinite clause as the Goal: it is (the negation of)
what we want to prove is logically implied by the truth of all the
definite clauses.  SLD resolution is called a backward chaining
strategy and Unit resolution is called a forward chaining strategy.
If we think of our Horn clauses as implications, we can see why: SLD
resolution starts from the Goal and reasons backwards using the
implications in a backward direction, from consequence to antecedent,
to determine what we have yet to establish in order to establish our
goal.  In Unit resolution we reason starting from facts we know (the
unit clauses) and reason forward to derive everything we can conclude,
eventually hoping to prove the Goal.

\section{First Order Logic (FOL)}

Propositional logic is very limited in what it can represent and
reason about.  Taking simple declarative sentences as primitive, and
thus not having the ability to look inside and analyze components of
simple sentences is very restrictive.  First order logic remedies this
by allowing primitive declarative sentences to be further decomposed
into statements about objects and relationships among objects.  For
(a classic) example, the sentence ``Socrates is a man'' in
propositional logic is primitive, but in first order logic we can
further analyze it and say, essentially, that Socrates is a thing in
the set of things called ``man'', written man(Socrates).  And we can
analyze the sentence ``All men are mortal'' as saying that if
something is in the set of men, then it is the set of mortal things,
written \verb|all(X,man(X) -> mortal(X))|.  The advantage of this is
that now we can reason about objects and their relationships.  For
example from these two first-order sentences, we can logically
conclude that Socrates is mortal, written mortal(Socrates).  In
propositional logic, this kind of reasoning can not be modeled.

\subsection{Syntax}

The syntax of FOL is:
\begin{verbatim}
term_list --> term, term_tail.

term_tail --> [].
term_tail --> [','], term_tail.

term --> [Atom], {fun_sym(Atom}}, opt_par_term_list.
term --> [X],{var_sym(X)}.

opt_par_term_list --> [].
opt_par_term_list --> ['('], term_list, [')'].

atomic_formula --> [Pred], {pred_sym(Pred)}, opt_par_term_list.

formula --> atomic_formula.
formula --> [not],formula.
formula --> formula, ['\/'], formula.
formula --> formula, ['/\'], formula.
formula --> [for_all],[X],{var_sym(X)},formula.
formula --> [there_exists],[X],{var_sym(X)},formula.
formula --> ['('], formula, [')'].
\end{verbatim}

In this prolog-like representation, we assume that the function
symbols (used in terms) come from some set of symbols, represented in
Prolog with the predicate \verb|fun_sym|.  Similarly predicate symbols
come from some fixed set as do variables, and the variable symbols
(called variables) and the function symbols are disjoint.

Assuming that the appropriate symbols are in the appropriate sets (and
variables are symbols starting with upper-case letters), examples of
terms are:
\begin{verbatim}
father(michael)
sum(1,3)
sum(5,prod(6,succ(3)))
\end{verbatim}
Examples of atomic formulas are:
\begin{verbatim}
man(socrates)
mortal(socrates)
employee(john,deere,93,124000,'101 Elm Street','Stony Brook','NY',11794)
loves(X,madonna)
\end{verbatim}
Examples formulas are:
\begin{verbatim}
mortal(socrates)
man(socrates) /\ mortal(socrates)
for_all X (not(man(X)) \/ mortal(X))
\end{verbatim}

We can define various syntactic properties of formulas that come in
handy.  

A variable occurrence is {\em free} in a formula if it does not appear
in the scope of a quantifier binding that same variable.  It is {\em
bound} otherwise.

A formula is {\em closed} if it contains no free occurrence of any
variable.  It is {\em open} otherwise.

\subsection{Semantics}

The semantics of FOL is given using FOL structures (sometimes called
models.)  A structure provides a meaning for each function symbol and
predicate symbol in the language.  Each function symbol and predicate
symbol comes with an ``arity'', a natural number that corresponds to
the number of terms that follow them in a parenthesized list in the
syntax.  (We assume, for simplicity now, each symbol has a fixed arity
and appears only in the context of that many following terms.)  A
structure comes with a domain of objects, say D.  A structure gives an
n-ary function on D as the meaning of a function symbol of arity n.  A
0-ary function symbol is called a constant and appears without any
following parenthesized list.  For example, ``socrates'' is a
constant.  So in particular, a structure assigns a constant to an
element of D.  A structure gives an n-ary relation on D as the meaning
of a predicate symbol of arity n.

We use structures to give meaning to first-order formulas.  But
first-order formulas contain variables, and we need a way to give
meanings to variables.  We do this with a ``variable assignment'', which
is an assumption of what values the variables take on.  A variable
assignment is a function from variables into the domain D of objects
of the structure, i.e., \verb|VA:Vars->D|.  Now given a structure and a
variable assigment VA, we can assign meanings to formulas of FOL.

We'll write a simple Prolog specification to show how meaning is defined.
A model M determines a domain, a function mapping and a predicate
mapping.  We represent that as the following Prolog facts:
\begin{verbatim}
%% given by the model
model_domain(Obj).                       % for every object Obj in the domain
model(FunctionSym,ArgList,Obj).          % to define every function symbol
model(PredicateSym,ArgList,TruthValue).  % to define every predicate symbol
\end{verbatim}
So for example, for a model of arithmetic, we would have facts like:
\verb|model_domain(0)|, \verb|model_domain(1)|,
\verb|model_domain(2)|,  and so on for the domain elements.  For
functions we would have \verb|model(+,[0,0],0)|,
\verb|model(+,[1,0],1)|, \verb|model(+,[1,1],2)| and so on.  For the
less-than predicate symbol, we would have \verb|model(<,[0,0],false)|,
\verb|model(<,[0,1],true)|, \verb|model(<,[1,0],false)|, and so on.
Of course, these sets are infinite, so we cannot actually write them
down as Prolog programs.  But the Prolog specification should be
clear; it's just that there would be infinitely many answers to some
queries.

Consider constant symbols in our arithmetic language.  Let's say that
we use boolean notation. So our model must also give a
meaning to these 0-ary function symbols making up boolean
representation of numbers.  That would be
done with the facts such as: \verb|model('11',[],3)| and
\verb|model('101',[],5)|.  We would also have the facts
\verb|model('0',[],0)| and \verb|model('1',[],1)| and it is important to
note that the first entry in these facts is a symbol, often called a
numeral, here a sequence of the 0 and 1 symbols.  The last element in
these facts is a number, the abstract concept regardless of what
notation we use to refer to it.

Now given that representation of a model, we can define the concept of
the meaning of a formula in a model under a variable assignment.  We
will assume that a variable assignment is represented as a list of
Var=Obj records.  First we have to define the meaning of a term in a
model under a variable assignment.  Such a meaning will be an object
in the model.  The definition is as follows:
\begin{verbatim}
%% meaning of a term in a model under a variable assignment
term_means(Term,VA,Obj) :- 
   var_sym(Term), 
   member(Term=Obj,VA).
term_means(Term,VA,Obj) :- 
   Term =.. [FunSym|Arguments], % e.g. f(a,X,g(b)) =.. [f,a,X,g(b)] 
   term_means_list(Arguments,VA,Objs),
   model(FunSym,[Objs],Obj).

%% meaning of a list of terms
term_means_list([],_,[]).
term_means_list([Term|Terms],VA,[Obj|Objs]) :-
    term_means(Term,VA,Obj),
    term_means_list(Terms,VA,Objs).
\end{verbatim}
So the meaning of a term is found by using the model to get the
meaning of constant symbols, using the variable assignment to get the
meanings of variables, and using the model's function definitions to
find the result of applying a function to its arguments.

Then given the meaning of terms, we can define the meaning of a
formula in a model under a truth assignment as follows:
\begin{verbatim}
%% meaning of a formula in a model under a variable assignment
formula_means(AFmla,VA,TVal) :-   % atomic formula
    Fmla =.. [PredSym|Args],      % e.g. psym(a,X,g(b)) =.. [psym,a,X,g(b)]
    term_means_list(Args,VA,Objs),
    model(PredSym,Objs,TVal).
formula_means(A /\ B,VA,TVal) :-
    formula_means(A,VA,TVal1),
    formula_means(B,VA,TVal2),
    /\(TVal1,TVal2,TVal).
formula_means(not(A),VA,TVal) :-
    formula_means(A,VA,TVal1),
    not(TVal1,TVal).
formula_means(exists(V,Fmla),VA,TVal) :-
    (model_domain(Obj),                   % try any domain object
     replace_assignment(V=Obj,VA,NewVA),  % for this variable
     formula_means(Fmla,NewvA,true)       % and if it makes subformula true
     ->	TVal = true
     ;  TVal = false
    ).

%% replace a single assignment with another
replace_assignment(V=Obj,VA,NewVA) :-
    append(Prefix,[V=_|Suffix],VA),
    append(Prefix,[V=Obj|Suffix],NewVA).

%% and truth table
/\(true,true,true).
/\(true,false,false).
/\(false,true,false).
/\(false,false,false).

%% or truth table
not(true,false).
not(false,true).
\end{verbatim}
To find the meaning of an atomic formula, we find the meanings of its
arguments using the term meaning function, and then use the model to
see whether that tuple of objects is in the relation that the model
associates with that predicate symbol.  The meaning of {\em and}
formulas is clear.  The meaning of an exists formula is true if we can
find some domain object in the model, that when we assign it to the
correspoding variable in a new variable assignment (leaving all other
individual assignments the same), that variable assignment makes the
subformula true.

(Do some examples.)

Properties of FOL formulas:

Two FOL formulas are {\em logically equivalent} if in every structure
and for every variable assignment, they have the same meaning.

The meaning of a closed formula (one with no free variables) is
independent of the variable assignment.  We normally work with closed
formulas, sometimes called sentences.

FOL equivalences (or definitions):
\begin{verbatim}
   not(forall(X,Fmla)) == exists(X,not(Fmla))
   not(exists(X,Fmla)) == forall(X,not(Fmla))
   forall(X,Fmla1) /\ forall(X,Fmla2) == forall(X,Fmla1/\Fmla2)
   if X is not free in Fmla2 then
      forall(X,Fmla1) \/ Fmla2 == forall(X,Fmla1\/Fmla2)  
   exists(X,Fmla1) \/ forall(X,Fmla2) == exists(X,Fmla1\/Fmla2)
   if X is not free in Fmla2 then
      exists(X,Fmla1) /\ Fmla2 == exists(X,Fmla1/\Fmla2)  
\end{verbatim}

Given a quantified formula such as \verb|forall(X,Fmla)|, we can choose a new
variable not appearing in \verb|Fmla|, say \verb|Y|, and replace every free
occurrence of \verb|X| in \verb|Fmla| with \verb|Y| getting \verb/Fmla|X<-Y/, and then
\verb/forall(Y,Fmla|X<-Y)/ is logically equvalent to \verb|forall(X,Fmla)|.

\subsection{Deduction}

We will again use resolution for our deduction method.  Recall that
resolution works on formulas in clausal form.  In FOL we will call an
atomic formula or a negation of an atomic formula a literal.  And
again a clause will be a set of literals.  Now a set of clauses is
true if they are true under all variable assignments.  A single clause is
true under a variable assignment if some literal in it is true under
that variable assignment.  This definition allows us to think of
the variables in clauses as being universally quantified.

So we would like to find, given an arbitrary FOL formula, an
equivalent clausal form.  This turns out not to be possible, but we
can find a clausal form that is satisfiable just in case the original
formula is satisfiable, and that is sufficient for our purpose of
deduction.

So given an arbitrary FOL formula, we first find a form in prenex
normal form (PNF), meaning all the quantifiers have outermost scope, and
the nonquantifier ``matrix'' is in conjunctive normal form.  For
example the formula:
\begin{verbatim}
forall(X,exists(Y,p(X) /\ (not(q(X) \/ r(X,Y)))))
\end{verbatim}
is in PNF.  Now given any FOL formula, we can find an equivalent one
in PNF.  First we ``standardize the variables apart'', which means
that we change bound variables to ensure that no two distinct
quantifiers in the formula bind the same variable.  This can always be
done by choosing new variables and changing one of the bound variables
to the new variable (including all free occurrences in the scope of
that binding quantifier).  Then we can first push all negations over quantifiers,
using the equivalences above, and over ``and'' and ``or'' using
DeMorgans laws (as for propositional logic), eliminating double
negations as necessary.  Next we have to pull quantifiers out of
conjunctions and disjunctions.  Given a formula of the form: 
\verb|(P /\ exists(X,Q))|, we can transform it to 
\verb|exists(X,P /\ Q)|, since X does
not appear in P (since all variables are standardized apart.)  And
similarly we can move all quantifiers out over ``and'' and ``or''.  So
for any FOL formula, we can find an equivalent formula in PNF.

For example, consider the formula for ``every man loves a woman'': \\
\verb|forall(X,man(X)->exists(Y,woman(Y)/\loves(X,Y)))|. We can find the
following chain of equivalences:
\begin{verbatim}
forall(X,man(X)->exists(Y,woman(Y)/\loves(X,Y)))
==> forall(X,not(man(X))\/exists(Y,woman(Y)/\loves(X,Y)))
==> forall(X,exists(Y,not(man(X)) \/ (woman(Y)/\loves(X,Y))))
==> forall(X,exists(Y,(not(man(X))\/woman(Y)) /\ (not(man(X))\/loves(X,Y))))
\end{verbatim}
and the final formula is in PNF.

Now if the formula in PNF has only universal quantifiers, then we are
done.  The ``matrix'' of the prenex formula directly provides the set
of clauses, since the assumption on the meaning of variables in a set
of clauses is that they are universally quantified over the entire
set.  But what happens if we have existential quantifiers in the
quantified prefix, as in the above example?

Here, we are not able to find a logically equivalent formula, in
general.  However we can find another formula that will be satisfiable
if and only if the PNF formula is satisfiable.  And for use in resolution
theorem proving that is just what we need.

So consider the PNF formula we have:
\begin{verbatim}
forall(X,exists(Y,(not(man(X))\/woman(Y)) /\ (not(man(X))\/loves(X,Y))))
\end{verbatim}
which says, ``for every X there is a Y such that some property
holds,'' in this case that Y is a woman that the man X loves.  So
let's consider a function ``lovee'' that for every man chooses a woman
that he loves.  So perhaps lovee(john)=mary where ``john loves mary''.
Now consider the formula:
\begin{verbatim}
forall(X,(not(man(X))\/woman(lovee(X))) /\ (not(man(X))\/loves(X,lovee(X)))))
\end{verbatim}
We have eliminated the existential quantifier on Y, and replaced every
occurrence of Y in the quantified formula with ``lovee(X)''.  The
claim is: there is a structure that satisfies the first formula if and
only if there is a structure that satisfies the second.  Consider the
``if'' direction and assume that there is a structure satisfying the
second formula.  Then that structure will satisfy the first formula,
since the witness for the truth of the exists subformula is given by
looking at where the ``lovee'' function takes the object that the
variable assignment assigns to X.  Since ``lovee'' is a function,
there will always be such an object, and the second formula ensures
that the subformula will be true using that object.  Now for the
``only if'' direction, the truth of the first formula guarantees that
there is a total function that takes any X to some Y satisfying the
subformula.  So there is a function, which can be assigned to the
``lovee'' function symbol in a new structure that will cause the
second formula to be satisfied.  

This is an example of a general transformation, called
``Skolemization''.  For any formula in PNF with an existential
quantifier in the prefix, we can eliminate the outermost existential
quantifier by introducing a new ``Skolem'' function symbol, with arity
equal to the number of universal quantifiers that include the
existential formula in their scope (i.e. to the left of the
existential.)  We then replace every occurrence of the existentially
quantified variable  with the term consisting of the new function
symbol applied to all the enclosing universally quantified variables.
This gives us another formula that is satisfiable if and only if the
original one is satisfiable.  By iterating this process, we can
eliminate all existential quantifiers and find a formula with only
universal quantifiers.  And that formula can now be used to generate
the equivalent set of clauses.

For example for our ``every man loves a woman'' formula, the set of
clauses is:
\begin{verbatim}
{(not(man(X)),woman(lovee(X))},
 {not(man(X)),loves(X,lovee(X))}
}
\end{verbatim}

\subsubsection{Resolution in FOL}

Resolution in FOL works with first-order clauses, i.e., sets of
first-order atom formulas and their negations.  So the clauses contain
variables.  We can informally think of a clause with variables as
being shorthand for the whole set of clauses one for each way to
substitute values for (i.e., instantiate) the variables in the clause
consistently.  In this way, we can think of the ``real'' clauses as
actually being variable-free in which case they are just propositional
clauses, (of course with the complication that there may be infinitely
many clauses.)  But our intuition should be that when we manipulate
these first-order clauses, we are intuitively manipulating their
(infinitely many) ground instantiations, i.e., instances with no
variables.

So as an example, consider the first-order clauses:
\begin{verbatim}
{loves(john,X),~woman(X)}
{~loves(Y,pat)}
\end{verbatim}
Thinking of these as representing their ground instances, we can see
that the first clause includes the instance
\verb|{loves(john,pat),~woman(pat)}| (since X can be instantiated
to \verb|pat|, and the second clause includes the instance
\verb|{~loves(john,pat)}| since Y can be instantiated to \verb|john|.
These (instantiated) clauses contain complementary literals and so we
can resolve them (as propositional clauses) to obtain a new clause:
\verb|{~woman(pat)}|.  And note that this fact does indeed follow from the
two first-order clauses.

Now consider a more complicated case:
\begin{verbatim}
{loves(Z,X),likes(Z,X)}
{~loves(Y,pat)}
\end{verbatim}
Now the first clause has an instance:
\verb|{loves(Z,pat),likes(Z,pat)}| which stands for many
ground clauses, one for each value substituted for Z.  And the second
clause \verb|{~loves(Y,pat)}| stands for many ground clauses, one
for each value substituted for Y.  Note that for any value, if we
substitute it for Z in the first and X in the second, we get two
resolvable clauses, and their resolvants are all the clauses we can
get by substituting something for Z in \verb|{likes(Z,pat)}|.  And
note that we can find this clause directly by matching loves(Y,pat)
of the second clause with loves(Z,X) of the first, to find the minimal
instantiations necessary to make them the same.  Then we can apply
those instantiations to all the literals in both clauses.  Then these
clauses can resolve directly generating a new first-order clause
representing all the ground resolvants.  This is the idea behind
first-order resolution.

There a a few things we must be careful about.  The matching operation
is called unification, and given two atomic formulas, it finds a most
general term that is an instance of both formulas.  This term is
called a most general unifier (mgu).  It turns out that given two
terms if an mgu exists, it is unique.   

There are several algorithms for finding the mgu.  One approach is to
work with a set of pairs of terms, which we try to unify.  If all
pairs are identical, we are done and the terms are unifiable.  If
there is a pair neither of whose components is a variable and whose
function symbols are not the same, then we are done and the terms are
not unifiable.  If one component of some pair is a variable and the
variable does not occur in the term it is paired with, but does appear
in some term in some other pair, then replace that variable by its
paired term in all terms in all other pairs in the set.  (Note that
this will eliminate all other occurrences of that variable.)  If there
are two nonvariable terms that are paired and have the same main
function symbol, then pair all the corresponding arguments of the two
terms and replace the original pair by that set of pairs.  Repeat
until done.  The most general unifier is determined by the remaining
set of pairs.  At least one of each final pair is a variable.

(More precise definition and give several examples)

Examples...

Two clauses are called {\bf variants} of each other if they are
identical up to renaming of variables.  Notice that the result of
replacing a clause in a set of clauses by a variant preserves its
meaning.  

Now we can define the first-order resolvant of two first-order
clauses.  Two clauses resolve if they contain complementary literals,
and the atomic formulas of those literals unify (after standardizing
their variables apart.)  The resolvent is obtained by taking the union
of the resolved clauses minus the resolved-upon literals and applying
the mgu substitution to that result.


So the resolution refutation algorithm for FOL is now the ``same'' as
the propositional algorithm.  Close the set of clauses under this
definition of resolution.  If the closure contains the empty clause,
then the original set is unsatisfiable.

Example:

Consider the example clauses, which we number for future reference:
\begin{verbatim}
1. { ap(n,X,X) }
2. { ap(c(X,Y),Z,c(X,W)), ~ap(Y,Z,W) }
3. { ~ap(c(a,c(b,n)),c(d,n),A) }
\end{verbatim}

Consider the following sequence of resolution steps to determine that
these clauses are inconsistent.
\begin{verbatim}
3. { ~ap(c(a,c(b,n)),c(d,n),A) }
resolve this with 2 using mgu:
    X = a, Y = c(b,n), Z = c(d,n), A = c(a,W)
obtaining:
4. { ~ap(c(b,n),c(d,n),W) }
resolve this with 2 again (with all variables primed) using mgu:
    X' = b, Y' = n, Z' = c(d,n), W = c(b,W')
obtaining:
5. { ~ap(n,c(d,n),W') }
resolve this with 1 using mgu:
    X = c(d,n), W' = c(d,n)
obtaining:
6. { }
\end{verbatim}

Since we derived the empty clause, these clauses are inconsistent.
Notice that we can look to see what instantiation of the variable A it
was that resulted in the inconsistency, by looking at the mgu's
involved:
\begin{verbatim}
A = c(a,W) from the first mgu
W = c(b,W') from the second, and
W' = c(d,n) from the last
yielding
A = c(a,c(b,c(d,n)))
\end{verbatim}
Now note that we can write clauses 1 and 2 in implicational form:
\begin{verbatim}
ap(Y,Z,W) -> ap(c(X,Y),Z,c(X,W))
ap(n,X,X)
\end{verbatim}
or in reverse implicational form, and expanding the predicate name of
{\em ap}:
\begin{verbatim}
append(c(X,Y),Z,c(X,W)) :- append(Y,Z,W).
append(nil,X,X).
\end{verbatim}
This is actually the append program, where we are representing the empty 
list with the constant nil, and using c/2 as the list constructor.

\subsubsection{Horn Clauses, SLD}

The definition of Horn clauses applies to first-order clauses: A
clause is a Horn clause if it contains at most one positive literal.
Resolution can be specialized to Horn clauses just as in the
propositional case, and for Horn clauses we specialize general
resolution to SLD resolution.  

Examples: 

\subsubsection{Prolog as FO Horn Clauses}
Prolog is depth-first search through the tree of SLD-resolution
proofs.

Consider:

\begin{verbatim}
{anc(X,Y), ~parent(X,Y)}
{anc(X,Y), ~parent(X,Z), ~anc(Z,Y)}
{parent(e,t)}
{parent(t,d)}
{parent(t,p)}
{parent(d,h)}
{parent(h,m)}
\end{verbatim}

This is the clause version of our old friend, the Prolog program for
defining ancestor.

Show the SLD tree, and Prolog evaluation.

\subsubsection{Limitations of Prolog/SLD and what can be done}

Prolog has infinite loops, as we've seen.  There are trivial Horn
clause programs that make perfect sense logically, but put SLD (and
thus Prolog) into an infinite loop. E.g. p :- p. But maybe more
importantly 
\begin{verbatim}
{anc(X,Y), ~parent(X,Y)}
{anc(X,Y), ~parent(X,Z), ~anc(Z,Y)}
\end{verbatim}
if parent is cyclic.  Or 
\begin{verbatim}
{anc(X,Y), ~parent(X,Y)}
{anc(X,Y), ~anc(X,Z), ~parent(Z,Y)}
\end{verbatim}
which logically is just as good as the previous definition.

Other natural examples from the previous chapter are: a) the
expression grammar and b) the simple English grammar as extended in
the exercises.  The expression grammar we gave has the wrong
associativity for the + and * operators; they are normally considered
to be left associative (e.g., 1+2+3 is considered as (1+2)+3 and not
1+(2+3)), but the grammar used right-associativity.  Had we used the
natural grammar for left associative operators, we would have had a
left-recursive grammar, and Prolog goes into an infinite loop for such
grammars.  Similarly for the simple English grammer in which we wanted
to say that a noun phrase is a noun phrase followed by a prepositional
phrase.  The natural rule to do this, verb|np --> np, pp.|, again
left-recursive.  So adding prepositional phrases in a way that Prolog
could handle them required some rather complicated, and perhaps
non-intuitive, machinations.


(Read intro now.)

Let's consider only Datalog programs, which are Horn clauses with only
constants and variables (i.e., no function symbols of arity greater
than 0.)

We can define an operator, called the $T_P$ operator, on sets of ground
atomic formulas as follows:

\begin{verbatim}
Tp(S) = {H : there is a ground instance, H :- B1,B2,...,Bn, of a rule,
             and for each Bi, Bi is in S}
\end{verbatim}

This is one-step of inference: if we assume all the facts of set S are
true, and the rules are true, then $T_P$(S) is what we can infer to be
true in one step.  We can infer what is true in any number of steps
beginning from no assumptions as follows:

\begin{verbatim}
S_0 = Tp([]).
S_1 = Tp(S_0).
S_2 = Tp(S_1).
...
S_n = Tp(S_n-1).
\end{verbatim}

And then we take the union of all the $S_i$.  This will give us all the
atomic formulas that must be true in every structure that makes all
the rules true.  This process essentially produces the ``least Model''
of the program.  For Datalog programs, this process is finite.
$S_i \subset S_{i+1}$ for every $i$, so the sequence is monotonically
increasing.  Since there are only finitely many possible ground atomic
formulas for a finite Datalog program, this sequence must eventually
reach a fixed point where $S_{k+1} = S_k$.

(Do ancestor example.)

Bottom-up computation starts with the facts and derives new facts
using the rules.  The idea is that you start assuming that you don't
know anything: i.e., that all the relations (corresponding to the rule
heads and facts) are empty.  Then you add tuples to those relations as
is required by the rules.  The simple example is:

Program:
\begin{verbatim}
e(a,b).
e(a,c).
e(c,d).
e(d,f).
p(X,Y) :- e(X,Y).
p(X,Y) :- p(X,Z),e(Z,Y).
\end{verbatim}

So there are two relations (sets of ordered pairs) defined by this
program: e/2 and p/2.  We start assuming they are empty:
\begin{verbatim}
e/2 = { }
p/2 = { }
\end{verbatim}
Now we use the current tuples in e/2 and p/2 (none, at the moment) to
see if these tuples and the rules and facts force us to add more
tuples.  Well, the 4 facts for e in the program force us to add those
pairs to the relation e/2.  So we get
\begin{verbatim}
e/2 = {<a,b>,<a,c>,<c,d>,<d,f>}
p/2 = { }
\end{verbatim}
Now we do it again, to see if we need to add more tuples, using these
current sets.  So since we have \verb|<a,b>| in e/2 and the first p-rule says
that p(X,Y) if e(X,Y), we must add \verb|<a,b>| to p/2.  And similarly for
the other pairs in e/2, so now we get:
\begin{verbatim}
e/2 = {<a,b>,<a,c>,<c,d>,<d,f>}
p/2 = {<a,b>,<a,c>,<c,d>,<d,f>}
\end{verbatim}
OK, do it again.  Now we have \verb|<a,c>| in p/2 and \verb|<c,d>| in e/2, and the
second p/2 rule says that p(X,Y) if p(X,Z) and e(Z,Y).  so with
X=a,Z=c and Y=d, we have to add \verb|<a,d>| to p/2.  And similarly for \verb|<c,d>|
in p/2 and \verb|<d,f>| in e/2, that same rule requires that we add \verb|<c,f>| to
p/2.  So at this point we get:
\begin{verbatim}
e/2 = {<a,b>,<a,c>,<c,d>,<d,f>}
p/2 = {<a,b>,<a,c>,<c,d>,<d,f>,<a,d>,<c,f>}
\end{verbatim}
And then again: now we have \verb|<a,d>| in p/2 and \verb|<d,f>| in e/2, so the
second p/2 rule requires that we add \verb|<a,f>| to p/2, and we get:
\begin{verbatim}
e/2 = {<a,b>,<a,c>,<c,d>,<d,f>}
p/2 = {<a,b>,<a,c>,<c,d>,<d,f>,<a,d>,<c,f>,<a,f>}
\end{verbatim}
and now we can't get anything new.  We just get pairs we already have,
so we have hit the fixed point, and are done.

The $T_P$ operator takes a current set of relations for the predicates
in the program, and uses the rules of the program P and the tuples in
the set of relations, and generates a new set of relations.  It does
this the way we did in each step above: Look for instantiated rules
where we have facts for all the atomic formulas on the right-hand side
of a rule, and then add the left-hand side.  So for example, what we
did above, in terms of the $T_P$ operator:

\begin{verbatim}
Tp(e/2={ },p/2={ })  =  T1  =  (e/2= {<a,b>,<a,c>,<c,d>,<d,f>}, p/2={ })
Tp(T1)  =  T2  =  (e/2= {<a,b>,<a,c>,<c,d>,<d,f>}, 
                   p/2= {<a,b>,<a,c>,<c,d>,<d,f>})
Tp(T2)  =  T3  =  (e/2= {<a,b>,<a,c>,<c,d>,<d,f>}, 
                   p/2= {<a,b>,<a,c>,<c,d>,<d,f>,<a,d>,<c,f>})
Tp(T3)  =  T4  =  (e/2= {<a,b>,<a,c>,<c,d>,<d,f>}, 
                   p/2= {<a,b>,<a,c>,<c,d>,<d,f>,<a,d>,<c,f>,<a,f>})
Tp(T4)  =  T4
\end{verbatim}
and we reached the fixed point.  Notice that we started with the empty
relations and at each step applied the %T_p% operator to the result of
the previous step.  This will always generate a non-decreasing
sequence of sets. If there are only finitely many possible tuples, it
will always stop at some point, and we will have reached a fixed
point.

Now if want to determine if a query is implied by a program, find the
LFP of the program using the $T_P$ operator, and see if the query is in
the LFP.  So this is another proof technique for Datalog programs.
(It is directly extensible to all Horn clause programs, but then the
LFP may be infinite.)

So this shows that for Datalog programs we don't {\em have} to live
with the infinite loops of Prolog.  There is an alternative.  

But there are problems with the $T_P$ approach: it is not goal-directed.
E.g., to find my ancestors, we have to find everybody's ancestors.
That can be a lot of unnecessary work.

We can see a similar (the same) phenonomenon with inductively defined
functions.  Recall from high school how we defined the factorial
function, n!

\begin{verbatim}
0! = 1
n! = n*(n-1)! if n>0
\end{verbatim}

There are two ways to think about evaluating, say 5!: We can think
``bottom-up'': 
\begin{verbatim}
0! = 1, 1! = 1*0!=1*1=1, 2! = 2*1!=2*1=2, 
3! = 3*2!=3*2=6, 4! = 4*3!=4*6=24, 5! = 5*4!=5*24=120.
\end{verbatim}

Or we can compute it ``top-down'' as a functional programming language
would by:
\begin{verbatim}
5! = 5*4! = 5*(4*3!) = 5*(4*(3*2!)) = 5*(4*(3*(2*1!))) = 5*(4*(3*(2*(1*0!)))) = 
5*(4*(3*(2*(1*1)))) = 5*(4*(3*(2*1))) = 5*(4*(3*2)) = 5*(4*6) = 5*24 = 120.
\end{verbatim}

For factorial, these two ways of computing it do essentially the same
amount of work.  But consider a definition of fibonacci:

\begin{verbatim}
fib(0) = 1
fib(1) = 1
fib(n) = fib(n-1) + fib(n-2) if n>1
\end{verbatim}

Computing this top-down takes exponential work.  (Do example)

Computing it bottom-up takes linear work. (Do example)

So it seems that bottom-up can be much better than top-down.  But
consider the definition:

\begin{verbatim}
lg(1) = 0
lg(n) = 1 + lg(n//2) if n>1. (where // is integer division)
\end{verbatim}

Computing this bottom up takes linear work, but computing it top-down
takes only log work.  So in this case top-down is much better than
bottom-up.  So it's clear that neither bottom up nor top-down is
uniformly better than the other.  The question is whether we can come
up with a single evaluation strategy that gets the best of both
strategies, for example would be linear on the given definition for
the fibonacci function and log on the definition of the lg function.

In the case of evaluting Datalog programs, there are two approaches
(which turn out to be essentially the same approach): Tabling, and
magic sets.  Tabling modifies Prolog backward chaining evaluation to
remember goals and answers to avoid rederiving redundant goals.  Magic
sets modifies the $T_P$ approach to add filtering predicates so that only
essential goals are derived.

We will look at the Tabling approach.

